\documentclass[footinclude=false,11pt,DIV11]{scrartcl}

\usepackage[english]{babel}
\usepackage[T1]{fontenc}
\usepackage[utf8]{inputenc}
\usepackage{graphicx}
\usepackage{amsmath}
\usepackage{mathtools}
\usepackage{array}
\usepackage{booktabs}
\usepackage{tabularx}
\usepackage{color}
\usepackage{colortbl}
\usepackage{listings}
\usepackage{enumerate}
\usepackage{upquote}
\usepackage[absolute]{textpos} % Manual placement of certain things
\usepackage{ragged2e} % Ragged-right columns with hyphenation
\usepackage{nicefrac}
\usepackage{macros}
\usepackage[format=hang,font=small,labelfont=bf]{caption}
\usepackage[expansion=false, babel=true]{microtype}
\usepackage{subfig}
\usepackage{ifthen}
\usepackage{longtable}
\usepackage{wrapfig}

% Make sure that ligatures remain searchable in the PDF
\input glyphtounicode
\pdfgentounicode=1

\IfFileExists{MinionPro.sty}
   {\usepackage[opticals,fullfamily,lf]{MinionPro}}
   {\usepackage{charter}\message{Package MinionPro.sty was not found. Reverting to
charter.}}

\setcounter{secnumdepth}{3}
\setcounter{tocdepth}{3}

\newcommand{\GuideVersion}{0.1.0}

\typearea[current]{last}
\raggedbottom
\renewcommand*\ttdefault{txtt}

\usepackage{scrpage2}
\ofoot[]{}
\cfoot[]{}
\automark[subsection]{section}
\ihead{\sc\leftmark}
\ohead{\sc\rightmark}
\chead{}
\setheadsepline{.2pt}
\setkomafont{pagenumber}{\normalfont}
\addtokomafont{sectioning}{\color{myblue}\rmfamily}
\addtokomafont{descriptionlabel}{\rmfamily}
\pagestyle{scrheadings}

\usepackage[
	bookmarks,
	bookmarksnumbered,
	colorlinks,
	plainpages=false,
	pdfpagelabels,
	hypertexnames=false,
	linkcolor=myblue,
	urlcolor=myblue,
	citecolor=myblue,
	pdfpagelabels,
	pdftitle={C/C++ Style Guide, version \GuideVersion},
	pdfauthor={Ioannis Gkioulekas},
	pdfstartview=FitH
]{hyperref}

\definecolor{myblue}{rgb}{0,.1,.6}
\definecolor{myred}{rgb}{0.63,.16,.16}
\definecolor{lstshade}{gray}{0.95}
\definecolor{lstframe}{gray}{0.80}
\definecolor{lstcomment}{gray}{0.5}
\definecolor{lstattrib}{rgb}{0,0.34,0}
\definecolor{remark}{rgb}{1.0, 0.9, 0.9}
\definecolor{remarkframe}{rgb}{1.0, 0.7, 0.7}

% Listings settings
\lstset{
	basicstyle = \small\ttfamily\raggedright,
	commentstyle=\color{lstcomment}\itshape,
	stringstyle=\color{lstattrib},
	mathescape = true,
	frame = lrtb,
	backgroundcolor = \color{lstshade},
	rulecolor = \color{lstframe},
	tabsize = 4,
	columns = flexible,
	keepspaces,
	belowskip = \smallskipamount,
	framerule = .7pt,
	breaklines = true,
	showstringspaces = false,
	keywordstyle = \bfseries,
	captionpos = b,
	upquote = true
}

\lstdefinelanguage{xml} {
	sensitive=true,
	morecomment=[s][\color{lstcomment}\itshape]{<!--}{-->},
	morecomment=[s][\color{lstcomment}]{<?}{?>},
	string=[b]", stringstyle=\color{lstattrib},
	keywords= [1] {
		shape,bsdf,scene,texture,phase,integer,float,
		string,transform,ref,rgb,srgb,spectrum,blackbody,
		medium,film,sampler,integrator,emitter,sensor,
		translate,rotate,scale,lookat,point,vector,matrix,
		include,fscat,volume,alias,rfilter,boolean,
		subsurface
	},
}


% Set up textpos
\TPGrid{68}{108}

% Thick frames for images
\setlength\fboxsep{0pt}
\setlength\fboxrule{1.5pt}

% Less vertical spacing for \figure[h] floats
\setlength{\intextsep}{3pt}

\lstnewenvironment{shell}[1][]{\lstset{#1}}
	{}
\lstnewenvironment{cpp}[1][]{\lstset{language=c++, #1}}
	{}
\lstnewenvironment{python}[1][]{\lstset{language=Python, #1}}
	{}
\lstnewenvironment{xml}[1][]{\lstset{language=xml, #1}}
	{}
\lstnewenvironment{console}[1][]{\lstset{basicstyle=\footnotesize\ttfamily, float, #1}}
	{}

\title{
	C/C++ Style Guide\\
	\LARGE Version \GuideVersion
}
\author{Ioannis Gkioulekas}
\date{\today}

\begin{document}
\maketitle

\tableofcontents


%%-----------------------------------------------------------------------------------
%% Introduction
\section{Introduction}

This guide documents the style conventions used in C/C++ libraries developed
by the author. \emph{Style} in this case refers to both source file formatting and
coding and design practices. These conventions are adopted with three objectives in
mind: readability, maintainability, and safety.

The adopted style conventions are strongly influenced by the Google C++ Style
Guide~\cite{Google} and the styling conventions of the Mitsuba Physically-Based
Renderer~\cite{Mitsuba}. With regards to the Google C++ Style Guide in particular, it
is assumed that its conventions apply by default, unless explicitly contradicted by a
convention described in this document. For this reason, the emphasis of this (evolving)
guide is on places where it contradicts the Google C++ Style Guide. In addition to
describing the adopted conventions, some effort has been put into also explaining the
rationale behind their use.

To help enforce consistency and application of its conventions, this guide is
accompanied by an Eclipse CDT-importable template that partially implements it.
Furthermore, given the large overlap with the Google C++ Style Guide,
\code{cpplint}~\cite{cpplint} can be used to identify violations of this guide's
conventions. Due to differences with the Google C++ Style Guide, it is recommended that
you call \code{cpplint} with the following arguments,
\begin{shell}
./cpplint.py --filter=-whitespace/tab,-whitespace/labels,-whitespace/comments
\end{shell}


%%-----------------------------------------------------------------------------------
%% Introduction
\section{General practices}

\paragraph{Enumerations:} Enumeration types should have a second to last \emph{length
entry} that is set equal to the number of valid entries (without the length entry), and
a last \emph{invalid entry} that is set equal to $-1$.

\paragraph{Boost:} Boost libraries should be used whenever this helps to save time or
write more compact code.

\paragraph{Argument order:} TODO.
% TODO: Should I deviate from Google's style and list outputs first?.


%%-----------------------------------------------------------------------------------
%% Formatting
\section{Formatting}

\paragraph{Indentation:} Indentation is done using tabs, which expand to \emph{four}
spaces.

\paragraph{Function declarations, definitions:} When wrapping a function's parameters
to stay within the 80-character line limit, parameters are aligned at either exactly
the column of the first function parameter when this can be achieved with only tabs; or
largest number of tabs that are not to the right of the first function parameter.

\paragraph{Function calls:} When wrapping a function's arguments to stay within the
80-character line limit, arguments are aligned at either exactly the column of the
first function arguments when this can be achieved with only tabs; or largest number of
tabs that are not to the right of the first function arguments.

\paragraph{Pointer and reference declarations:} The asterisk and ampersand are placed
adjacent to the type.

\paragraph{Class format:} The \code{public}, \code{protected}, and \code{private}
keywords are at the same column as the \code{class} keyword.


%%-----------------------------------------------------------------------------------
%% Naming
\section{Naming conventions}

\paragraph{File names:} Header files end in \code{.h}, and C/C++ implementation files
in \code{.cpp}.

\paragraph{Name format:} Names are always written in camel-case, perhaps with the
addition of a prefix consisting of lower-case characters followed by an underscore,
and with the exception of macros. There are described in detail below.

\paragraph{Rule priority:} TODO.
% TODO: Derive and write rule priority.

\paragraph{Classes and structures:} Classes and structures use camel-case
starting with a capital letter.

\paragraph{Type names:} Type names use camel-case starting with a capital letter.

\paragraph{Variables:} Variables use camel-case starting with a lower-case letter.

\paragraph{Functions:} Functions use camel-case starting with a lower-case letter.

\paragraph{Macros:} Macros (which should be used scarcely) use all-capital letters
with underscores for separating words.

\paragraph{Enumerations:} Enumeration types and entries start with a capital
\textbf{E} (\Etype). The length entry has the name of the corresponding enumeration
type with the suffix \code{Length}, and the invalid entry has the name of the
corresponding enumeration type with the suffix \code{Invalid}. Variables that are of a
type of an enumeration start with \textbf{e} (\etype). This is to enhance
recognizability of these types and variables.
% TODO: What about enums used as constants?

\paragraph{Bools:} Functions returning \code{bool} and variables of type \code{bool}
start with \code{is} (\btype). This is to enhance recognizability of these variables
and functions.

\paragraph{Constants:} Compile-time constants start with \textbf{k} (\ktype). This is
to enhance recognizability of these variables.

\paragraph{Pointers:} Type names that are defined as pointers to some other type start
with a capital \textbf{P} (\Ptype). Variables that are of type pointer to some
type start with \textbf{p} (\ptype). This is to enhance recognizability of these types
and variables. Note that a variable that is of a \Ptype type is not a \ptype variable.

\paragraph{Attributes:} Attributes of classes and structs have the prefix \code{m\_}.
This is to enhance recognizability, but also to facilitate maintenance, by making
exploring the attributes of a class straightforward (e.g., type \code{m\_}, and
auto-complete).

\paragraph{Accessors and mutators:} Accessor (``getter'') and mutator (``setter'')
member functions of a class are named by concatenating the prefix \code{get\_} and
\code{set\_}, respectively, and the part of the name of the attribute they provide
access to after its \code{m\_} prefix. This is to enhance recognizability, but also to
facilitate maintenance, by making exploring functions that provide access to attributes
of a class straightforward (e.g., type \code{get\_} or \code{set\_}, and
auto-complete). In mutators, the parameter is named after the attribute it is assigned
to, without the \code{m\_} prefix.

\paragraph{Constructors:} In copy constructors, the parameter is named after the class
with the first letter in lower-case. In other constructors, function parameters are
named after the attributes they are assigned to, without the \code{m\_} prefix.

\paragraph{Operators:} In assignment operators, the parameter is named after the class
with the first letter in lower-case.
% TODO: Derive and write rules for other operators.

\paragraph{Static and const:} When a variable is declared both \code{static} and
\code{const}, the \code{static} keyword should be the leftmost one.


%%-----------------------------------------------------------------------------------
%% Example
\section{Example}

Below is a comprehensive example demonstrating many of the above conventions.
% TODO: Test that example compiles.
\begin{cpp}
#define NUM_DAYS_OF_YEAR 360

typedef enum EMonth {
	EJanuary = 0,
	EFebrurary,
	/* ... */
	EDecember,
	EMonthLength,
	EMonthInvalid = -1
} EMonth;

static const int kNumOfDays = 30;

struct Date {
public:
	int m_day;
	EMonth m_eMonth;
	int m_year;
}

bool isValidDate(const Date& query);

typedef Date* PDate;

class LogEntry {
public:
	LogEntry(const PDate& entryDate, 
				const double* pEntryData)
				: m_entryDate(entryDate),
				  m_pEntryData(pEntryData) { }

	LogEntry(const LogEntry& logEntry);

	LogEntry& operator=(const LogEntry& logEntry);

	inline const PDate& get_entryDate const {
		return m_entryDate;
	}

	inline const double* get_pEntryData const {
		return m_pEntryData;
	}

	inline void set_entryDate(const PDate& entryDate) {
		m_entryDate = entryDate;
	}

	inline void set_pEntryData(const double* pEntryData) {
		m_pEntryData = pEntryData;
	}

	void printLogEntry();

private:
	PDate m_entryDate;
	double* m_pEntryData;
};
\end{cpp}


%%-----------------------------------------------------------------------------------
%% Bibliography
\bibliographystyle{acm}
\bibliography{cpp_style}

\end{document}
